\documentclass[a4paper,10pt,landscape]{article}
\usepackage[top=3mm,bottom=3mm, left=6mm,right=6mm]{geometry}
\usepackage{graphics}
\usepackage{fancybox}
\usepackage{framed}
\usepackage{multicol}
\usepackage{xltxtra}
\usepackage{zxjatype}
%\setjamainfont{Noto Sans CJK JP}
\setjamainfont{Source Han Sans}
\pagestyle{empty}
\begin{document}
\centering
\scalebox{0.92}{
\begin{tabular}{|c|l||c|l||c|l|}
\hline \texttt{a} & あ! \ あー  \ あはは!  & \texttt{lete} & 寒さ \ 冷たい \ 凍る \ 調理されていない  & \texttt{pi} & 名詞句に属す「の」  \\
\hline \texttt{akesi} & 可愛くない動物  & \texttt{li} & 動詞句の分離  & \texttt{pilin} & 気持ち  \  感情  \  心臓  \  感銘  \  感じる  \  考える  \\
\hline \texttt{ala} & 否定 \ 0 \ いいえ  & \texttt{lili} & 小さい  \  少ない  \  若い  \  脆い  \  減らす(cf.mute)  & \texttt{pimeja} & 黒  \  影  \  暗い(cf.walo)  \\
\hline \texttt{alasa} & 集める \ 狩る  & \texttt{linja} & 紐状のもの  & \texttt{pini} & 終わり  \  端  \  終わる(cf.open)  \  過去(cf.kama \ sin)  \\
\hline \texttt{ale/ali} & 全て  & \texttt{lipu} & 薄く平で折り曲げられるもの  \  紙  \  ウェブサイト  & \texttt{pipi} & 虫  \  昆虫  \  蜘蛛  \\
\hline \texttt{anpa} & 下 \ 底(cf.sewi)  & \texttt{loje} & 赤  & \texttt{po} & †4 \\
\hline \texttt{ante} & 違い \ 違う \ 変える(cf.pona \ awen) \ あるいは  & \texttt{lon} & 存在 \ (場所)に,で  & \texttt{poka} & 側面  \  隣(cf.weka)  \  次  \  腰  \  一緒に  \  ともに  \\
\hline \texttt{anu} & または \ あるいは  & \texttt{luka} & 手 \ 腕 \ 5  & \texttt{poki} & 入れ物  \  容器  \  箱  \  コップ  \\
\hline \texttt{apeja} & †恥 \ 罪  & \texttt{lukin} & 見る  \  読む  \  注目する  \  視覚的に  \  みため(cf.kute)  & \texttt{pona} & よい  \  簡単  \  単純  \  直す  \  いいね!  \\
\hline \texttt{awen} & 待つ \ 保つ \ 残る \ 不変(cf.ante)  & \texttt{lupa} & 穴状のもの  \  出入り口  \  ドア  \  窓  & \texttt{powe} & †架空の,偽の,騙す \\
\hline \texttt{e} & 目的語の分離  & \texttt{ma} & 場所  \  土地  \  国  \  領域  & \texttt{pu} & 本(主に公式Toki Pona本)  \\
\hline \texttt{en} & 列挙の「と」  & \texttt{majuna} & †古い cf.sin ala \  pini  & \texttt{sama} & 同じ  \  等しい  \  似た  \  〜のような  \\
\hline \texttt{epiku} & *すごい! \ 最高! & \texttt{mama} & 親  & \texttt{seli} & 火  \  暖かい  \  調理する  \\
\hline \texttt{esun} & 売る \ 買う \ 取引する \ 市場  & \texttt{mani} & お金  \  富  & \texttt{selo} & 外側  \  表面  \  皮(cf.insa)  \\
\hline \texttt{ijo} & もの \ こと \ 物体  & \texttt{meli} & 女  & \texttt{seme} & 何  \  どれ  \  誰  \  疑問代名詞  \\
\hline \texttt{ike} & 悪い \ 悪 \ 悲しいかな  & \texttt{mi} & 私  \  私の  & \texttt{sewi} & 高さ  \  上(cf.anpa)  \  信仰心のある  \  礼儀正しい  \\
\hline \texttt{iki} & †彼 \ 彼女 \ それ(cf.ona) & \texttt{mije} & 男  & \texttt{sijelo} & 体  \  体調  \\
\hline \texttt{ilo} & 道具 \ 装置 \ 機械  & \texttt{moku} & 食べ物  \  食事  \  食べる  \  飲む  \  消費する  & \texttt{sike} & 円  \  ボール  \  車輪  \  回る  \  丸い  \\
\hline \texttt{insa} & 間 \ 中間  & \texttt{moli} & 死  \  死ぬ  \  殺す  & \texttt{sin} & 新しい  \  新鮮  \  修復(cf.pona)  \  他の  \  もっと(cf.mute)  \\
\hline \texttt{jaki} & 汚れ \ ゴミ  & \texttt{monsi} & 後ろ  \  背中(cf.sinpin)  & \texttt{sina} & あなた  \  あなたの  \\
\hline \texttt{jan} & 人 \ 人間らしい  & \texttt{monsuta} & †怪物  \  恐怖  \  恐れる(cf.akesi)  & \texttt{sinpin} & 垂直面(cf.supa)  \  前  \  胸  \  胴(cf.monsi)  \  壁  \\
\hline \texttt{jelo} & 黄色 \ 萌黄色  & \texttt{mu} & にゃーん  \  動物の鳴き声  & \texttt{sitelen} & 絵  \  イメージ  \  描く  \  書く  \  デザインする  \\
\hline \texttt{jo} & 持つ \ 含む  & \texttt{mulapisu} & *ピザ & \texttt{soko} & *きのこ \ 菌 \\
\hline \texttt{kala} & 魚 \ 海の生き物  & \texttt{mun} & 月(cf.suno)  & \texttt{sona} & 知識  \  知性  \  理解  \  知る  \  わかる  \  学ぶ  \\
\hline \texttt{kalama} & 音 \ 音を出す  & \texttt{musi} & 遊び  \  遊ぶ  \  楽しむ  & \texttt{soweli} & かわいい動物  \  ≡╹ω╹≡  \\
\hline \texttt{kama} & 出来事 \ 来る \ なる \ 未来  & \texttt{mute} & 量  \  多い  \  とても  \  豊富な  \  増やす(cf.lili)  & \texttt{suli} & 大きい  \  大きさ  \  高い  \  大人  \  大切  \\
\hline \texttt{kan} & †一緒に \ ともに(cf.poka)  & \texttt{namako} & 余分  \  不要(cf.wile)  \  装飾  \  スパイス  \  塩  & \texttt{suno} & 太陽(cf.mun)  \  光  \\
\hline \texttt{kapa} & †膨れた形のもの(山,ボタンなど)(cf.nena) & \texttt{nanpa} & 数  \  番号  \  番目  & \texttt{supa} & 水平面(cf.sinpin)  \  台  \  机  \  椅子  \\
\hline \texttt{kapesi} & †茶色,灰色 & \texttt{nasa} & 狂った  \  酔った  \  変わった  & \texttt{suwi} & 甘い物  \  甘い  \\
\hline \texttt{kasi} & 植物 \ 葉 \ 木  & \texttt{nasin} & 道  \  方法  \  習慣  & \texttt{tan} & 起源  \  原因  \  〜から  \  〜のために  \\
\hline \texttt{ken} & できる \ 許す \ 可能性  & \texttt{nena} & 山  \  鼻  \  出っ張り  \  ボタン  & \texttt{taso} & ただ  \  のみ  \  しかし  \\
\hline \texttt{kepeken} & 使う \ 用いて  & \texttt{ni} & この  \  その  & \texttt{tawa} & 行く  \  動き  \  動かす  \  〜へと  \  〜にとって  \\
\hline \texttt{kijetesantakalu} & *アライグマ  & \texttt{nimi} & 名前  \  単語  & \texttt{telo} & 水  \  液体  \  飲み物  \  濡らす  \  洗う  \\
\hline \texttt{kili} & 野菜 \ 果実 \ きのこ  & \texttt{noka} & 足  \  脚  & \texttt{tenpo} & 時  \  時間  \  瞬間  \  状況  \\
\hline \texttt{kin} & もまた \ さらに \ やはり  & \texttt{o} & ねえ  \  おい  \  呼格の分離  \  命令法  & \texttt{toki} & 言語  \  会話  \  話す  \  こんにちは  \\
\hline \texttt{kipisi} & †切る(cf.tu)  & \texttt{oko} & 目  \    & \texttt{tomo} & 家  \  建物  \\
\hline \texttt{kiwen} & 石 \ 固い \ 鉱物 \ 個体の  & \texttt{olin} & 愛  \  愛する  & \texttt{tonsi} & *LGBT+ \\
\hline \texttt{ko} & 半個体 \ 粉状糊状のもの  & \texttt{ona} & 彼  \  彼女  \  それ  & \texttt{tu} & 2  \  ペア  \  分ける  \  二倍にする  \\
\hline \texttt{kokosila} & *トキポナ以外の言語で話す & \texttt{open} & 開く  \  起動  & \texttt{tuli} & †3 \\
\hline \texttt{kon} & 空気 \ 風 \ 匂い \ 精神 \ 気体の  & \texttt{pakala} & 失敗  \  壊す  \  傷つける  \  なんてこった  & \texttt{unpa} & 性  \  性交  \\
\hline \texttt{kule} & 色 \ 塗る \ いろとりどり  & \texttt{pake} & †止める  \  阻止する  & \texttt{uta} & 口  \  胃  \\
\hline \texttt{kulupu} & グループ \ コミュニティ \ 共有の  & \texttt{pali} & 活動  \  計画  \  仕事  \  作る  \  建てる  \  働く  \  機能する  & \texttt{utala} & 戦い  \  競争  \  不調和(cf.olin)  \  攻撃  \\
\hline \texttt{kute} & 聞く \ 聴く \ 聴覚的に(cf.lukin)  & \texttt{palisa} & 棒状のもの  & \texttt{walo} & 白  \  明るい(cf.pimeja)  \\
\hline \texttt{la} & 副詞句の分離  & \texttt{pan} & 主食  \  パン  \  ご飯  & \texttt{wan} & 1  \  要素  \  まとめる  \  あわせる  \\
\hline \texttt{lape} & 寝る \ 休む  & \texttt{pana} & 与える  \  送る  \  交換  & \texttt{waso} & 鳥  \\
\hline \texttt{laso} & 青 \ 青緑  & \texttt{pasila} & †良い,簡単(cf.pona) & \texttt{wawa} & 力  \  強い  \  速い  \  声が大きい  \\
\hline \texttt{lawa} & 頭 \ 思考 \ 率いる \ 導く  & \texttt{pata} & †兄弟(cf.jan sama)  & \texttt{weka} & 遠い  \  不在(cf.lon)  \  捨てる  \  外す  \\
\hline \texttt{leko} & 四角 \ ブロック  & \texttt{peto} & *泣く(Pana E Telo Oko) & \texttt{wile} & ほしい  \  したい  \  必要(cf.namako)  \  希望  \\
\hline \texttt{len} & 布 \ 服 \ 着る  & \texttt{peta} & *緑 & \texttt{yupekosi} & *過去の創作を台無しにする \\
\hline
\end{tabular}
}
\rightline{\footnotesize{†: 古い語, *: 新しい語}\hspace{5em}}
\end{document}
